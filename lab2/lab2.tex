\documentclass[a4paper,12pt]{article}

\usepackage[T2A]{fontenc}
\usepackage[utf8]{inputenc}
\usepackage[english, russian]{babel}

\usepackage{amsthm, amsmath, amssymb}

\usepackage{graphicx}

\usepackage{hyperref}
\usepackage{float}

\usepackage[a4paper, total={6.5in, 10in}]{geometry}


\begin{document}
% Title page 
\begin{titlepage}
    \begin{center}
        \textsc{
            Санкт-Петербургский политехнический университет имени Петра Великого \\[5mm]
            Физико-механический институт\\[2mm]
            Высшая школа прикладной математики и физики            
        }   
        \vfill
        \textbf{\large
            Отчет по лаборабороной работу №2\\
            по дисциплине "Интервальный анализ"\\[3mm]
        }                
    \end{center}

    \vfill
    \hfill
    \begin{minipage}{0.5\textwidth}
        Выполнил: \\[2mm]   
		Студент: Наветкина Ирина \\
		Группа: 5030102/00201\\
    \end{minipage}

	\hfill
	\begin{minipage}{0.5\textwidth}
		Принял: \\[2mm]
		к. ф.-м. н., доцент \\   
		Баженов Александр Николаевич
	\end{minipage}

    \vfill
    
    \begin{center}
    Санкт-Петербург\\
    2023 г.\\
    \end{center}
\end{titlepage}

\tableofcontents
\newpage

\section{Постановка задачи}
Имеем ИСЛАУ:
\begin{equation*}
 \begin{cases}
   [2, 4]*x_1 + [4,6]*x_2 = [5, 9]\\
   3*x_1 + [-6, -4]*x_2 = [-1, 1]\\
   [0.5, 1.5]*x_1 = [1, 3]\\
  [0.5, 1.5]*x_2 = [0, 2]\\
 \end{cases}
\end{equation*}
Необходимо найти решения ЛЗД. Для этого нужно
\begin{itemize}
    \item исследовать разрешимость ЛЗД (найти максимум распознающего функционала)
    \item коррекция ЛЗД
    \begin{itemize}
        \item достижение разрешимости ИСЛАУ за счет коррекции правой части (равномерное/неравномерное)
        \item достижение разрешимости ИСЛАУ за счет коррекции матрицы
    \end{itemize}
\end{itemize}


\section{Теория}
\subsection{Распознающий функционал}
Разпознающим функционалом называется функция
\begin{equation}
    \mathrm{Tol}(x)=\mathrm{Tol}(x,\mathbf{A},\mathbf{b})=\min_{1\leq i\leq m}\left\{\mathrm{rad}(\mathbf{b}_i)-\left|\mathrm{mid}(\mathbf{b}_i)-\sum_{j=1}^n \mathbf{a}_{ij}x_j\right|\right\}
\end{equation}
Пусть
\begin{equation}
T=\max_{x\in \mathbb{R}^n} \;\mathrm{Tol}(x, \mathbf{A},\mathbf{b})
\end{equation}
и это значение достигается распознающим функционалом в некоторой точке $\tau\in \mathbb{R}^n$. Тогда
\begin{itemize}
    \item если $T\geq0$, то $\tau\in\Xi_{\mathrm{tol}}(\textbf{A},\textbf{b})\neq  \emptyset$, т.е. линейная задача о допусках для интервальной линейной системы $\textbf{A}x=\textbf{b}$ совместна и точка $\tau$ лежит в допусковом множестве решений.
    \item если $T>0$ то $\tau\in int \;\Xi_{\mathrm{tol}}(\textbf{A},\textbf{b})\neq  \emptyset$, и принадлежность $\tau$ допусковому множеству решений устойчива к малым возмущениям данных - матрицы и правой части.
    \item если $T<0$ то $\Xi_{\mathrm{tol}}(\textbf{A},\textbf{b})=\emptyset$,  т.е. линейная задача о допусках для интервальной линейной системы $\textbf{A}x=\textbf{b}$ несовместна.
\end{itemize}
\subsection{Достижение разрешимости ИСЛАУ путём изменения правой части}
\textbf{Равномерное уширение правой части
ИСЛАУ}\\
Расширение вектора $\textbf{b}$ происходит путем его замены на вектор: 
\begin{equation}
    \textbf{b}+K\textbf{e}, \;\; K\geq 0, \;\; \textbf{e}=([-1, 1],...,[-1, 1])^T
\end{equation}
Тогда
\begin{equation}
\max_{x\in \mathbb{R}^n} \;\mathrm{Tol}(x, \mathbf{A},\mathbf{b}+K\textbf{e}) = T + K
\end{equation}
Но $\mathrm{Arg}\max\mathrm{Tol}$ - не изменится(положение точки Т) \\
\\
\textbf{Неравномерное уширение правой части
ИСЛАУ}\\
Если линейная задача о допусках с матрицей $\textbf{A}$ и вектором правой части $\textbf{b}$ первоначально не имела решений, то новая задача с той же матрицей  $\textbf{A}$ и уширенным вектором
\begin{equation}
    (\textbf{b}_i+K\cdot v_i\cdot [-1,1])_{i=1}^m
\end{equation}
в правой части становится разрешимой при $K\geq |T_v|$, где
\begin{equation}
    T_v=\min_{1\leq i\leq m}\left\{v_i^{-1}\left(\mathrm{rad}(\mathbf{b}_i)-\left|\mathrm{mid}(\mathbf{b}_i)-\sum_{j=1}^n \mathbf{a}_{ij}x_j\right|\right)\right\}
\end{equation}
Значение $\mathrm{Arg}\max\mathrm{Tol}$ -  изменится
\subsection{Достижение разрешимости ИСЛАУ путём изменения матрицы}
Общая схема равномерного метода заключается в том, что необходимо модифицировать матрицу $\textbf{A}$ засчет ее замены на $\mathbf{A}\ominus \mathbf{E}$, где 
% $N = \{v_i\}$ - матрица весов $K$ - общий коэффициент сужения $\mathbf{A}, \; 
$\mathbf{E}$ состоит из $\mathbf{e}_{ij}=[-e_{ij}, e_{ij}]$ \\
Причем значения точечных величин $\mathbf{e}_{ij}$ удовлетворяют двум условиям:
\begin{equation}
    0 \leq \mathbf{e}_{ij} \leq rad{\;\mathbf{a}_{ij}} \\
\end{equation}
\begin{equation}
    \sum_{j=1}^n e_{ij}\tau=K, \quad i=1,2,...,m, \quad K>0
\end{equation}
Если $K\geq |T|$, то тогда линейная задача о допусках с матрицей $\mathbf{A}\ominus\mathbf{E}=([\underline{\mathbf{a}}_{ij}-\underline{\mathbf{e}}_{ij},\overline{\mathbf{a}}_{ij}+\overline{\mathbf{e}}_{ij}])$ и правой частью $\textbf{b}$ становится разрешимой.
\subsection{Оценки вариабельности решения}
Абсолютной вариабельностью оценки называется величина
\begin{equation}
    \mathrm{ive}(\mathbf{A},\mathbf{b})=\min\limits_{A\in\mathbf{A}}\mathrm{cond}\:A\cdot||\argmax\:\mathrm{Tol}(x)||\frac{\max\limits_{x\in\mathbb{R}^n}\mathrm{Tol}(x)}{||\mathbf{b}||}
\end{equation}
Относительной вариабельностью оценки называется величина
\begin{equation}
    \mathrm{rve}(\mathbf{A},\mathbf{b})=\min\limits_{A\in\mathbf{A}}\mathrm{cond}\:A\cdot\max\limits_{x\in\mathbb{R}^n}\mathrm{Tol}(x)
\end{equation}

\section{Результаты}
\subsection{Максимум распознающего функционала}
\begin{figure}[H]
\centering
\includegraphics[width=0.7\textwidth]{Tol1.png}
\caption{Расположение максимума распознающего функционала} 
\end{figure}
Максимум со значением $T=-0.333$ расположен в точке $\tau=(1.333,0.6667)$.
\subsection{Достижение разрешимости за счёт коррекции правой части}
\subsubsection{Равномерное уширение}
Положим $K=1$. Получается максимум со значением $T=0.6667$ расположен в точке $\tau=(1.333,0.6667)$. При этом видим, что точка максимума осталась прежней.
\begin{figure}[H] \label{MatrixCorrSet}
    \centering
    \includegraphics[width=0.7\textwidth]{even_right.png}
    \caption{Допусковое множество решений с равномерным уширением правой части} 
\end{figure}
\noindent На данном рисунке черный квадрат имеет сторону $2\cdot \mathrm{ive}$, а фиолетовый - $2\cdot \mathrm{rve}$. Область обведенная черной линией - допусковое множество\\
$\mathrm{ive}: 0.1473502 \;\;\;\mathrm{rve}: 0.726363348$

\subsubsection{Неравномерное уширение}
Выберем произвольный вектор $v = [0.5, 0.1, 1, 0.5]$ и постоянную $K = 3$. \\
Получается максимум со значением $T=0.724$ расположен в точке $\tau=(0.96, 0.576)$. При этом видим, что точка максимума сдвинулась.
\begin{figure}[H] \label{MatrixCorrSet}
    \centering
    \includegraphics[width=0.7\textwidth]{uneven_right.png}
    \caption{Допусковое множество решений с неравномерным уширением правой части} 
\end{figure}
\noindent На данном рисунке черный квадрат имеет сторону $2\cdot \mathrm{ive}$, а фиолетовый - $2\cdot \mathrm{rve}$. Область обведенная черной линией - допусковое множество\\
$\mathrm{ive}: 0.106272696 \;\;\;\mathrm{rve}: 0.69755409$

\subsection{Достижение разрешимости за счёт коррекции левой части}
Для построения интервальной матрицы $\textbf{E}$ с уравновешанными интервальными элементами $\mathbf{e}_{ij}=[-e_{ij}, e_{ij}]$ выбираем такие $\mathbf{e}_{ij}$, что:
\begin{equation*}
 \begin{cases}
    0 \leq \mathbf{e} \leq 1 = \mathrm{rad}(\textbf{a}_{11}), \mathrm{rad}(\textbf{a}_{12}), \mathrm{rad}(\textbf{a}_{22}) \\
    0 \leq \mathbf{e} \leq 0.5 = \mathrm{rad}(\textbf{a}_{31}), \mathrm{rad}(\textbf{a}_{42}) \\
    1.333*\mathbf{e} + 0.666*\mathbf{e} = K \geq |T| = 0.333
 \end{cases}
 \Rightarrow 0.1666 \leq \mathbf{e} \leq 0.5
\end{equation*}
сформируем матрицу 
 \begin{equation}
\math{E}=
\begin{pmatrix}
[-0.5, 0.5] & [-0.4, 0.4] \\
[-0.3, 0.3] & [-0.5, 0.5] \\
[-0.5, 0.5] & 0 \\
0 & [-0.3, 0.3] \\
\end{pmatrix}
\end{equation}
\begin{figure}[H] \label{MatrixCorrSet}
    \centering
    \includegraphics[width=0.7\textwidth]{left.png}
    \caption{Допусковое множество решений со скоректированной матрицей} 
\end{figure}
\noindent На данном рисунке черный квадрат имеет сторону $2\cdot \mathrm{ive}$, а фиолетовый - $2\cdot \mathrm{rve}. $ Область обведенная черной линией - допусковое множество\\
$\mathrm{ive}: 0.068321425 \;\;\;\mathrm{rve}: 0.3444088$
\begin{figure}[H] \label{MatrixCorrSet}
    \centering
    \includegraphics[width=0.6\textwidth]{left_bliz.png}
    \caption{Допусковое множество решений со скоректированной матрицей} 
\end{figure}


\section{Вывод}
\begin{enumerate}
    \item После корекции правой части равномерным уширением вектора $\textbf{b}$ безусловный максимум распознающего функционала $Tol = 0.667$  в точке (1.333, 0.667), а  форма поверхности распознающего функционала $Tol$ не изменилась. \\
    Для нерасномерного уширения видно, что формы поверхности и положения безусловных максимумов распознающего функционала $Tol$ до и после неравномерного уширения вектора $\textbf{b}$ не совпадают. \\
    \item Сравнивая данные коррекции можно сказать, что неравномерное уширение дает меньшие значения вариабельности по сравнению с равномерным.
    \item Квадрат $rve$ на нашем примере не очень хорошо приближает допусковое множество решений, а вот квадрат $ive$ почти полностью принадлежит $\Xi_{\mathrm{tol}}$. И это очевидно, ведь мы имеем хорошую обусловленность матрицы $(\mathrm{cond}(\mathrm{mid}\,\mathbf{A})=1.638)$.
\end{enumerate}

\end{document}